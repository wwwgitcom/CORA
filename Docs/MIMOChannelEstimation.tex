\documentclass{article}

\usepackage[no-math]{fontspec}
\usepackage{zhspacing}
\zhspacing
%\usepackage{zhfont}
%\usepackage{zhmath}
%\setzhmainfont{Microsoft YaHei}
%\setzhmonofont{Microsoft YaHei}
\setmainfont[Mapping=tex-text]{TeX Gyre Termes}
\usepackage[top=1in, bottom=1in, left=1in, right=1in]{geometry}
\usepackage{indentfirst}
\setlength{\parindent}{22pt}

%\usepackage{balance}
%\usepackage{cite}
%\usepackage{times}
%\usepackage{amsmath}
\usepackage{graphicx}
\usepackage{graphics}
\usepackage{epsfig}
%\usepackage{latexsym}
\usepackage{amsfonts}
%\usepackage{amssymb}
%\usepackage{paralist}
\usepackage{xspace}
%\usepackage{mathrsfs}
%\usepackage{amssymb}
\usepackage{color}
\usepackage{algorithm}
\usepackage{algorithmic}
\usepackage{listings}
%\usepackage{multirow}
\usepackage{booktabs}
%\usepackage{ulem}

%
\usepackage[
pdfauthor={Ji Fang},
pdftitle={4x4 MIMO Channel Estimation},
pdfkeywords={MIMO, Channel estimation, Wireless},
pdfborder={0 0 0},
colorlinks=true
]{hyperref}

\begin{document}
\title{4x4 MIMO Channel Estimation}
\author{Ji~Fang}

\maketitle


\section{4x4 MIMO 信道估计}

802.11n中4x4 Preamble正交序列
\begin{table}[tbh]
\centering
\begin{tabular}{c|c|c|c|c}
\hline
 & t1 & t2 & t3 & t4 \\
\hline
P1 & 1  & -1 & 1 & 1 \\
P2 & 1  & 1 & -1 & 1 \\
P3 & 1  & 1 & 1 & -1 \\
P4 & -1  & 1 & 1 & 1 \\
\hline
\end{tabular}
\caption{802.11n HTLTF 正交化序列}
\label{tab:dot11nHTLTFOtho}
\end{table}

接收端第一根天线在各时刻接收到的信号为(X为HTLTF序列)
\begin{eqnarray}
y_{1, t1} & = & h_{11}X + h_{12}X + h_{13}X - h_{14}X \\
y_{1, t2} & = & -h_{11}X + h_{12}X + h_{13}X + h_{14}X \\
y_{1, t3} & = & h_{11}X - h_{12}X + h_{13}X + h_{14}X \\
y_{1, t4} & = & h_{11}X + h_{12}X - h_{13}X + h_{14}X
\end{eqnarray}
解方程
\begin{eqnarray}
y_{1} & = & y_{1, t1} + y_{1, t4} = 2h_{11}X  + 2h_{12}X  \\
y_{2} & = & y_{1, t2} -  y_{1, t3} = -2h_{11}X  + 2h_{12}X   \\
y_{3} & = & y_{1, t1} - y_{1, t4} = 2h_{13}X  - 2h_{14}X  \\
y_{4} & = & y_{1, t2} + y_{1, t3} = 2h_{13}X  + 2h_{14}X  \\
h_{11} & = & (y_{1} - y_{2}) / 4X \\
h_{12} & = & (y_{1} + y_{2}) / 4X \\
h_{13} & = & (y_{3} + y_{4}) / 4X  \\
h_{14} & = & (y_{4} - y_{3}) / 4X
\end{eqnarray}
其余信道可由上述方法解出。
各天线$i$可并行计算各自的信道$h_{i, j}, j \in \{1, 2, 3, 4\}$


\end{document}
