
\section{分布式频偏估计}

假设有M个发送端(STA),N个接收端(AP),各自的中心频率不相同。
设$\Delta f_{ij}$为$STA_{j}$与$AP_{i}$之间的频偏。

在802.11a/g/n中,使用legacy long training field(L-LTF)来估计$\Delta f_{ij}$。
$\Delta f_{ij} = -\frac{arg[ y(t - \delta t)y^{*}(t)  ]}{2 \pi \delta t}$, $\delta t = 3.2 \mu s$。$arg$被定义为$arctan$,其取值范围为$(-\frac{\pi}{2}, +\frac{\pi}{2})$。所以利用L-LTF估计的频偏范围为$(-78.125KHz, +78.125KHz)$。
子载波间隔为$\Delta f = \frac{20MHz}{64} = 312.5KHz$。

在时域接收到的信号为$y(t)=x(t) e^{j2 \pi \Delta f t}$,在频域为$Y = X(f - f')$。即时域旋转等价于频域移位。

粗频偏、细频偏应该是纠偏过程之中的概念,感觉两者没有什么实质差别,看的无非是一个纠偏精确度的问题;其实可以把粗频偏和细频偏合到一起做也可以的。当然纠偏也可以采用第一次纠偏之后再跟踪残余频偏的方式进行。
\begin{itemize}
\item 整数倍频偏和小数倍频偏是频偏大小的概念。一般用好的天线是不存在整数倍频偏的。整数倍频偏导致的是子载波数据的偏移,而小数倍频偏导致的是ICI。
\item 不能先进行整数倍频偏估计再进行小数倍频偏估计。因为一般整数倍频偏导致的是子载波数据的偏移,其估计方式是在频域估计;而小数倍频偏会导致ICI,使得频域数据无法提取。所以必须要先估计出小数倍频偏,消除了ICI,提取出频域子载波数据,然后再做整数倍频偏估计。
\end{itemize}

发送的时域信号为$x=[x_{0}, x_{1}, \cdots, x_{63}]$,设频偏为$\Delta f$,则接收到的时域信号为
\begin{eqnarray}
r&=&[x_{0}e^{j\frac{2\pi*0\Delta f}{f_{s}}}, x_{1}e^{j\frac{2\pi*1\Delta f}{f_{s}}}, \cdots, x_{63}e^{j\frac{2\pi*63\Delta f}{f_{s}}}]
\end{eqnarray}
接收到的频域信号为$y=FFT(r)$。











以2x2为例,每个用户频域上有4个pilot,用户i的pilot分别为p1,p2,p3,p4,接收端收到的信号为,接收到的信号为y1,y2,y3,y4
\begin{eqnarray}
y_{1} & = & h_{11}f_{11}p_{11} + h_{12}f_{12}p_{21} \\
y_{2} & = & h_{11}f_{12}p_{12} + h_{22}f_{22}p_{22} \\
y_{3} & = & h_{11}f_{13}p_{13} + h_{22}f_{22}p_{23} \\
y_{4} & = & h_{11}f_{14}p_{14} + h_{22}f_{22}p_{24} \\
\end{eqnarray}
p1和p2为已知,解上述方程可得$f_{11}$和$f_{}$

